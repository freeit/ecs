\chapter{ECS}\index{ECS}
The elearning community server (ECS) is designed as a message oriented
middleware (MOM) and is implemented as a
\href{http://en.wikipedia.org/wiki/REST}{REST} conform application.

Because the ECS was born in an elearning context the following definition shows
consideration of that. Nevertheless the ECS could be used in other areas of
responsibility.

The ECS groups its participants in so called \gq{communities}. Participants
could address each other only if they share a community. Therefore they could
address an explicit participant, a participant list or the whole members of the
community (see ECS API for details).

All participants have to register at the ECS. Every registered participant has
access to at least three system resources (\rs{/system/memberships},
\rs{/system/events}, \rs{/system/auths}) to get informed and take part at a ECSA
network. To design/map your specific application communication you can create as many
application resources you want.

\hypertarget{selfrouting}{}
\section{Community selfrouting}%
\index{selfrouting}%
\index{ECS!selfrouting}
If community selfrouting is activated at the participant (administration area)
you can decide if you also want to receive the message which you send to an
appropriate community, i.e. you get an event notification (if events on this
resource is activated) and you get it listed by its
\href{https://ecs.uni-stuttgart.de/ecsa-wiki/HeikoBernloehr/FreeLancer/ECS/ecs2/core#list-ressource}{list
resource} and could access it through its
\href{https://ecs.uni-stuttgart.de/ecsa-wiki/HeikoBernloehr/FreeLancer/ECS/ecs2/core#queue-ressource}{queue
resource}. Of course, as sender of the message you can always access it by its
\href{https://ecs.uni-stuttgart.de/ecsa-wiki/HeikoBernloehr/FreeLancer/ECS/ecs2/core#message-resource}{message
resource}. 

\hypertarget{authentication}{}
\section{Authentication}%
\index{authentication}%
\index{ECS!authentication}
All participants have to be authenticated in order to use ECS services. A
participant is deemed to be authenticated if the \hv{X-EcsAuthId} header is set and
the ECS knows it. The real authentication take place in front of the ECS,
normally at the Webserver (Proxy). But this depends on
configuration/installation of ECS:
\begin{txtg}
   +-----------+    .        
   |   ECS     |   /_\       
   | (RAILS)   |    |        
   +-----------+    |        
   |Rack Module|    |        
   | (optional)|    |        
   +-----------+    | Message
   | Webserver |    | Flow   
   |  (Proxy)  |    |        
   +-----------+    |        
         |          |        
   +-----------+    |        
   |Participant|    |        
   +-----------+             
\end{txtg}
Currently supported authentication methods:
\begin{itemize}
  \item Basic Auth
  \item X.509 certificates
\end{itemize}

\hypertarget{anonymous-participants}{}
\subsection{Anonymous participants}%
\index{anonymous-participants}%
\index{ECS!anonymous participants}
The creation of a new anonymous participant automatically takes place by every
call to an ECS resource if the calling participant didn't set \hv{X-EcsAuthId}
or \hv{Cookie} header, by setting a \hv{Set-Cookie} header in the response. On
subsequent calls the participant has to provide this cookie in a \hv{Cookie}
header in order to be identified as the previously calling participant.
Additionally those participants were automatically joined to the \kw{public}
community. Further their lifetime will be limited and all resources will be
silently deleted after this lifetime becomes zero. With succesional accesses to
ECS this lifetime will be refreshed. For general cookie handling see also
\href{http://en.wikipedia.org/wiki/HTTP\_cookie}{HTTP cookies}. See also
\href{https://ecs.uni-stuttgart.de/ecsa-wiki/HeikoBernloehr/FreeLancer/ECS/ecs2/core/examples}{examples}. 



\hypertarget{system_resources}{}
\section{System resources}%
\index{system resources}%
\index{ECS!system resources}

\hypertarget{events_resource}{}
\subsection{Events}%
\index{events resource}%
\index{ECS!events resource}
Provides a general queue which accumulates the resource tasks: creation,
deletion and renewal. Available representations are application/json and
application/xml. It's recommended to use the events queue to supervise all your
possible application specific resources. Further you only have to poll the
events queue in order to supervise all your application specific ressources and
this further take down system load. 

Remark: If you wisely decide to use the events queue to supervise your
application specific ressources you have to manage the validity of events queue
yourself, i.e. you shouldn't additionally poll your application specific
ressources directly, because then you will get stale events in the events
queue. 
\begin{description}
  \item[/sys/events] GET provides a list of events for the appropriate calling
  participant. Optionally the query string parameter \textit{count} could be used
  to limit the amount of returned events. 
  \item[/sys/events/fifo] GET provides an event (the oldest one) for the
  appropriate calling participant. Optionally the query string parameter
  \textit{count} could be used to extend the amount of returned events.  POST
  provides an event (the oldest one) for the appropriate calling participant
  and removes it from the events queue. Optionally the query string parameter
  \textit{count} could be used to extend the amount of returned events.
\end{description}
Following a sample representation in JSON:
\begin{code}
[
  {
    "status": "created",
    "ressource": "numlab/exercises/7"
  },
  {
    "status": "destroyed",
    "ressource": "numlab/exercises/3"
  }
]
\end{code}

\hypertarget{memberships_resource}{}
\subsection{Memberships}%
\index{memberships resource}%
\index{ECS!memberschips resource}
Provides information of the affiliation of the calling participant to the
available communities. Available representations are \hv{application/json} and
\hv{application/xml}.
\begin{description}
  \item[/sys/memberships] GET provides a list of memberships for the
  appropriate calling participant. It implies all participants joining an
  appropriate community including the caller itself.
\end{description}
With the \kw{itsyou} key the caller of the \rs{/sys/memberships} resource will be
informed which participant in the different communities is assigned to him.

Following a sample representation in JSON:
\begin{code}
[
  {
    "community": {
      "name": "cc_courselinks",
      "description": "CampusConnect courselinks."
    },
    "participants": [
      {
        "name": "ILIAS-ECS Client 1",
        "itsyou": true,
        "org": {
          "name": "Leifos",
          "abbr": "LEI"
        },
        "mid": 1,
        "description": "Development participant.",
        "dns": "n/a",
        "email": "meyer@leifos.com"
      },
      {
        "name": "FreeIT.de Testparticipant",
        "itsyou": false,
        "org": {
          "name": "FreeIT Softwaredevelopment.",
          "abbr": "FreeIT"
        },
        "mid": 2,
        "description": "A general test participant.",
        "dns": "n/a",
        "email": "Heiko.Bernloehr@FreeIT.de"
      },
      {
        "name": "ILIAS-ECS Client 2",
        "itsyou": false,
        "org": {
          "name": "Leifos",
          "abbr": "LEI"
        },
        "mid": 3,
        "description": "",
        "dns": "n/a",
        "email": "meyer@leifos.com"
      }
    ]
  }
]
\end{code}

\hypertarget{auths_resource}{}
\subsection{Auths}%
\index{auths resource}%
\index{ECS!auths resource}
This means authorization through one touch tokens. Provides a mechanism to
grant each participant authorization to consume services from any
service-providing-participant in an ECS network.

The interface is the same as for
\href{https://ecs.uni-stuttgart.de/ecsa-wiki/HeikoBernloehr/FreeLancer/ECS/ecs2/core#application\_resources}{application
resources}. If you want to create an authorization token, you have to provide
at least a json formated URL (authorization context) as body text:
\begin{code}
curl ... -X POST -d '{"url":"https://freeit.de/mycourse"}' https://.../sys/auths
\end{code}
and you will get back something like this:
\begin{code}
{
  "hash": "5a944e72346e6e3102d32ccfecc18862d23e1dc0",
  "sov": "2011-03-08T23:25:27+01:00",
  "eov": "2011-03-08T23:26:27+01:00",
  "url": "https://freeit.de/testcourse",
  "abbr": "LEI"
}
\end{code}
\begin{description}
  \item[hash] povides the authorization token (one touch token) 
  \item[sov] stands for start of validation 
  \item[eov] stands for end of validation 
  \item[url] provides the authorization context 
  \item[abbr] provides an abbreviation of the participant which has been
  created the authorization token (deprecated) 
\end{description}
You're allowed to set the \kw{sov} and/or \kw{eov} to distinguish the validity
period of the authorization token. If you do not, the validity period is set to
one minute starting at current time.

The recommended way to fetch an authorization token when knowing the one touch hash:
\begin{code}
curl .... -X DELETE https://.../sys/auths/<one touch hash>
\end{code}
This will return the auths representation (same structure/form as when
creating; see above) and delete it server side. If the authorization token is
outtimed, i.e. the current time is not between \kw{sov} and \kw{eov}, you will
get back a return code 409 (conflict) and following descriptional text in the
body: \textit{Authorization token outtimed}.
