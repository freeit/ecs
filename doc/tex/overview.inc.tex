\chapter{Overview}

An ECSA\nc{ECSA}{elearning community service architecture}\index{ECSA!overview}
is a service architecture for elearning based webservices. It provides
mechanisms for communication and authorization between elearning systems among
each other and management systems. This is implemented via a
MOM\nc{MOM}{message orientated middleware}.

The ECSA is derived from an architecture style for distributed systems
called REST. 

It will put major efforts that only recognized Web standards/protocols
and web components are used, whereby a high degree of compatibility
and connectivity is achieved. See figure \ref{fig:ecsa_components}
for ECSA components.

\begin{figure}
\noindent \begin{centering}
\includegraphics{../diagrams/ecsa_overview}
\par\end{centering}
\caption[ECSA components.]{\label{fig:ecsa_components}Components of an ECSA network.}
\end{figure}

An ECSA builds up of three primary components:
\begin{itemize}
  \item The ECS (elearning community server)\nc{ECS}{elearning community
  server} serves the core functionionality of an ECSA network. It provides
  named message resources to allow communication between all participants. 
  \item An ECC (elearning community client)\nc{ECC}{elearning community client}
  is a participant\nc{participant}{a client in ECSA network} in an ECSA
  network. It has to be registered at ECS and must be able to talk to the ECS
  as a REST based client. This participant normally has a native implementation
  of the ECS interface. Our favourite ECCs are LMSs (learning management
  systems).
  \item An ECP (elearning community proxy)\nc{ECP}{elearning community proxy}
  represents a special kind of participant. It serves as a proxy for a none
  ECSA compliant system so that such a system is able to participate in an ECSA
  network without ever knowing about it. 
\end{itemize}

\section{Sample usage scenario}

Suppose you have several LMSs (learning management systems)\nc{LMS}{learning
management system} and want to share courses between them. You decide not
to interchange the real courses but only course links which consist of some
meta data of the appropriate course especially a link formed by an URL pointing
to the real course so you can call it through the WWW e.g.: 

\verb!http://ilias.freeit.de/goto.php?target=pg_26_43&client_id=ecs2}!

Now it's possible for each LMS to communicate the released courses by the
resources provided from the ECS to an explicit LMS (point to point) or to a
community of LMSs (point to multipoint). 

Because of the uniform application interface -- there are only GET,
PUT, DELETE and POST operations -- receiving participants can fetch
messages through a GET on the resource URL or sending messages
by a POST on the resource URL (with some additional query parameters
or header variables to point to the appropriate receivers).

To illustrate this we use the simple ECC application \cmd{curl} to
send a message from one participant to another:

\begin{verbatim}
curl -i -H 'X-EcsAuthId: pid01' \
        -H 'X-EcsReceiverMemberships: mid02' \
        -H 'Content-Type: application/json' \
        -X POST \
        -d '{
              "name": "Mathematics II",
              "url" : "http://ilias...?target=pg_26_43&client_id=ecs2",
              ...
            }' \
        http://ecs.freeit.de/campusconnect/courselinks
\end{verbatim}
In order to receive a message (in fifo mode) the receiving
participant may call:
\begin{verbatim}
curl -i -H 'X-EcsAuthId: pid02' \
        -H 'Accept: text/plain; application/json' \
        -X GET \
        http://ecs.freeit.de/campusconnect/courselinks/fifo
\end{verbatim}

Of course, there are several ways to operate on a resource. For details
on using the resources located on an ECS and the different parameters
(http headers, query strings) please see XXX for details.

